% NIH Grant Proposal for the Specific Aims and Research Plan Sections
%----------------------------------------------------------------------------------------
%	PACKAGES AND OTHER DOCUMENT CONFIGURATIONS
%----------------------------------------------------------------------------------------

\documentclass[11pt,notitlepage]{article}

% A note on fonts: As of 2013, NIH allows Georgia, Arial, Helvetica, and Palatino Linotype. LaTeX doesn't have Georgia or Arial built in; you can try to come up with your own solution if you wish to use those fonts. Here, Palatino & Helvetica are available, leave the font you want to use uncommented while commenting out the other one.

\usepackage[super]{natbib}
\usepackage{palatino} % Palatino font
%\usepackage{helvet} % Helvetica font
\renewcommand*\familydefault{\sfdefault} % Use the sans serif version of the font
\usepackage[T1]{fontenc}
\linespread{1.05} % A little extra line spread is better for the Palatino font
\usepackage{hyperref} % to allow hyperlinks to websites on the internet
\usepackage[hypcap]{caption} % to point to the top of the image
\usepackage{lipsum} % Used for inserting dummy 'Lorem ipsum' text into the template
\usepackage{amsfonts, amsmath, amsthm, amssymb} % For math fonts, symbols and environments
\usepackage{graphicx} % Required for including images
\usepackage{booktabs} % Top and bottom rules for tables
\usepackage{wrapfig} % Allows in-line images
%\usepackage[labelfont=\small]{caption} % Make figure numbering in captions bold
\usepackage[top=0.6in,bottom=0.6in,left=0.6in,right=0.6in]{geometry} % Reduce the size of the margin

\usepackage{booktabs}
\usepackage{graphicx}
\usepackage[table,xcdraw]{xcolor}
\pagestyle{empty} % Remove page numbers
%\let\oldtabular\tabular 
%\renewcommand{\tabular}{\large\oldtabular}

\hyphenation{ionto-pho-re-tic iso-tro-pic fortran} % Specifies custom hyphenation points for words or words that shouldn't be hyphenated at all

  % to reduce white space between SECTIONS
\usepackage[compact]{titlesec}
\titlespacing{\part}{0pt}{*0pt}{*0pt}
\titlespacing{\section}{0pt}{*0}{*0}
\titlespacing{\subsection}{0pt}{*0}{*0}
\titlespacing{\subsubsection}{0pt}{*0}{*0}

% Reduce whitespace arount lists
\usepackage{mdwlist}
\usepackage{verbatim}
%\usepackage[compact]{titlesec}
%\titlespacing{\part}{-200pt}{-100pt}{-200pt}
%\titlespacing{\section}{0pt}{0pt}{0pt}
%\titlespacing{\subsection}{0pt}{*0}{*0}
%\titlespacing{\subsubsection}{-2pt}{*0}{*0}
%\titlespacing{\subparagraph}{-5pt}{*0}{*0}
%\titlespacing*{\subparagraph} {\parindent}{1ex plus 1ex minus .2ex}{0.5em}

  
  % to reduce white space between PARAGRAPHS
%\setlength{\parskip}{-2pt}
% \setlength{\parsep}{-2pt}

  % additional parameters
\setlength{\headsep}{20pt}
%\setlength{\topskip}{0pt}
%\setlength{\topmargin}{0pt}
%\setlength{\topsep}{0pt}
\setlength{\partopsep}{-2pt}
%\setlength{\parindent}{1cm}

% to reduce white space around figures
% \setlength{\textfloatsep}{0pt plus 0pt minus 0pt}

% Set graphics path to tell latex where to look for images
%\graphicspath{ {C:/Users/Micheal/Dropbox/Professional/KL2/BigDataR01/ConceptGraphics/} }

\begin{document}
\subsection*{Specific Aims}

National health databases and electronic health records (EHR) are 
clustered by procedure, provider, service, institution and geography and inherently incomplete. Often 
neglected, this rich spatial and temporal organization is most realistically 
captured in hierarchical statistical models with cluster-specific parameters, but 
estimating such models is still difficult for medical researchers using traditional software.
Uncollected, incomplete or missing data can further hinder modeling or bias inferences.
We propose to further develop the capabilities of the software called \textit{Stan} --- 
which is a probabilistic programming language, mathematical library, suite of estimation
algorithms, and ecosystem of supporting interfaces --- in order to make advanced hierarchical  
modeling accessible to data scientists for EHR-based medical outcomes research with incomplete data.
Making \textit{Stan} more accessible has the potential to transform the way medical outcomes research is conducted.

\paragraph*{Flexibility and robustness of hierarchical modeling could 
transform EHR based outcomes research.} Consider surgery as an 
illustrative example. Patients in the same hospital undergoing the 
same surgical intervention by the same team will show similar clinical 
trajectories and responses. We are interested in investigating
differences in therapeutic effects and in predicting adverse outcomes in order to prevent 
them. Doing so entails (1) estimating individual intercept-shifts for each provider and procedure to control for potentially confounding differences 
in quality of care by different teams, (2) allowing for spatial clustering of adherence  behavior, e.g., by different services, 
which can be represented by multilevel modeling, and (3) partially pooling estimates to improve precision, especially in subgroups with sparse data.
Prediction of adverse health outcomes can be improved by exploiting the implied correlations between different but related subsets of data. 
Conversely, failure to account for the highly structured and correlated nature of health care delivery or 
failure to account for the mechanisms by which some data are missing may lead to incorrect statistical inferences, poor predictions, 
and adverse health consequences. Realistic modeling of the heterogeneity in care delivered may help to identify reasons for variance 
in performance and may point to ways to improve outcomes. 

\paragraph*{Classical approaches and traditional software often lack flexibility 
for hierarchical modeling.} Most available software limits the types of 
hierarchical models that can be estimated because their algorithms are more likely 
to run into computational problems when estimating more complicated hierarchical models.
In contrast, \textit{Stan} is a flexible, general-purpose modeling language and a 
novel, powerful estimation engine that has facilitated advanced hierarchical modeling in biostatistics, epidemiology, public health, 
political science, and pharmacokinetics. \textit{Stan's} development has been funded by the NSF, DoD, and other organizations. 
\textit{Stan} and the interfaces to it (e.g. from \textit{Python, Julia, R}, etc.) are open source and platform-independent.

\paragraph*{Good hierarchical modeling should be accessible, transparent and requires good model diagnostics} 
However, building and tuning sophisticated hierarchical models with \textit{Stan} (or other available software) 
is still very challenging even for the initiated. We lack visual exploratory and diagnostic tools 
to recognize when a complicated model is logically  flawed or fails to fit the data empirically. We propose to enhance \textit{Stan's} 
ecosystem with more accessible interfaces and statistical tools for principled model checking and re-specification.

\paragraph*{Aim 1:} To further develop our software package \textit{rstanarm}, a more user-friendly interface to \textit{Stan}, 
for the open-source statistical language and environment \textit{R}, in order to make it more suitable and accessible for clinical data 
scientists and biostatisticians who want to estimate hierarchical models for EMR data.

\paragraph*{Aim 2:} To further develop \textit{shinystan}, our interactive web application to 
analyze and visually explore the output of \textit{Stan} models and to develop diagnostics 
to identify and troubleshoot computational and empirical problems with advanced hierarchical models.

\paragraph*{Aim 3:} To further develop \textit{mi}, our \textit{R} package for multiple imputation
of missing data so that it can make use of \textit{Stan} via the models provided by \textit{rstanarm}.

\paragraph*{Aim 4:} To explicate, document, and disseminate realistic hierarchical models for incomplete data 
to the clinical data science community with hands on use cases, workshops, journal articles, and 
online tutorials. To solicit the data scientist community feedback, engage new 
software developers, and to incorporate improvements to our software.


\end{document}
